\chapter*{Introducción}\label{chapter:introduction}
\addcontentsline{toc}{chapter}{Introducción}

    Desde hace varios años la Inteligencia Artificial (IA) viene revolucionando e \\impactando significativamente en muchas esferas de la 
vida del hombre. Dentro de la IA uno de los campos que más actividad tiene es el Procesamiento de Lenguaje Natural (PLN). Este abarca el conjunto de 
t\'ecnicas computacionales que tienen como objetivo el trabajo con el lenguaje humano, que van desde la extracci\'on de entidades en textos hasta modelos 
de comprensi\'on y generación de textos. Tanto la generación de texto a texto (\emph{text-to-text}, en inglés) como la generación de 
datos a texto (D2T por sus siglas en inglés, \emph{data-to-text} ) son instancias de la Generación de Lenguaje Natural (GLN). Reiter y Dale \brackcite{Reiter1997BuildingAN} 
caracterizan la GLN como el subcampo de la IA y la lingüística computacional que se ocupa de la construcción de sistemas informáticos que pueden 
producir textos comprensibles a partir de alguna representación no lingüística subyacente de la información. 
Esta definición se adapta fácilmente a los sistemas cuya entrada consiste en datos y es asumida en este trabajo para referirse a los sistemas de GLN. 
    
    
    Existe consenso en la forma en que la salida de un sistema de GLN debe presentarse: texto. Sin embargo, no hay establecido un est\'andar en cuanto a la forma en 
que se presentan los datos para su procesamiento, variando de un sistema a otro \brackcite{reiter_dale_2000,Gatt2018SurveyOT}. Como regla general el texto producido por estos sistemas 
debe mantener fidelidad a los datos que lo originan y debe ser consecuente con su intención comunicativa \brackcite{reiter_dale_2000}, no siendo lo mismo un 
sistema para la generación de diálogos que uno que tiene como objetivo describir resúmenes biográficos. Propuestas tempranas de sistemas como Ana \brackcite{kukich1983design} para 
generar reportes financieros, o como FoG \brackcite{goldberg1994using}, generador de reportes climáticos, comenzaron a mostrar las capacidades de estos 
modelos a la hora de dotar de interpretabilidad y relevancia datos que se presentaban de forma repetitiva y tediosa.
  
    En un contexto donde la producción de datos se ha acelerado como resultado de los avances tecnológicos y la digitalización de los sistemas industriales, se ha hecho necesario para 
las empresas el manejo y la interpretación de los mismos. Por esa raz\'on, algunas empresas se han especializado y han comenzado a brindar estos servicios a otras \brackcite{dale2020natural}. 
Un ejemplo es la compañía \textit{Automated Insights\footnote[1]{https://automatedinsights.com/}} que ha enfocado su negocio en brindar soluciones a otras corporaciones con vista a automatizar sus 
procesos de producción de texto. Entre los casos de uso clásico de estas soluciones encontramos, en el marco del comercio electr\'onico, la generación de descripciones de 
productos a partir de sus fichas técnicas. El periodismo ha sido otra de las esferas beneficiadas de estos avances en la GLN. La generación automática de noticias (conocida como 
periodismo robótico) está cada vez más extendida, al punto de que importantes editoriales como \textit{The Washington Post} han creado sus propios sistemas para la generación de 
texto a partir de datos\footnote[2]{https://www.washingtonpost.com/pr/wp/2017/09/01/the-washington-post-leverages-heliograf-to-cover-high-school-football/}. 
En este caso, su sistema se apoda \textit{Heliograf} y les permite cubrir todos los partidos de fútbol americano de las escuelas secundarias del área de Washington DC 
cada semana.\\

    \textbf{Motivaciones}\\

    A partir de las perspectivas que se abren en el campo de la GLN, surge la motivación del presente trabajo. Siendo el deporte un campo 
que despierta tanto interés y que es fuente de entretenimiento de muchas personas, sentar las bases del diseño de un sistema para la generación de resúmenes en español de 
eventos deportivos es un reto estimulante. Por su carácter estadístico y su gran audiencia, el deporte es una de las esferas que se puede beneficiar claramente del desarrollo de los sistemas 
de generación automática de texto. Muchos trabajos se han enfocado en este tema \brackcite{theune2001data, van2017pass, gunasiri2021automated}. Y aunque es relativamente sencillo encontrar en la web 
los datos o tablas estadísticas de un determinado enfrentamiento deportivo, la gran variedad de eventos que se suceden constantemente hacen 
que sea humanamente imposible darle cobertura a cada uno de ellos a nivel de narración y resumen. Por esta razón es motivante desarrollar  sistemas que sean capaces de cubrir muchas esferas del deporte.\\
Asimismo, este trabajo se presenta en el marco de las líneas de investigación existentes en el grupo de Inteligencia Artificial de la Facultad de Matemática y Computación de la 
Universidad de La Habana (MATCOM) y puede servir de base para futuros trabajos que sigan ampliando y profundizando en la GLN.\\


    \textbf{Antecedente}\\

    La intención de adentrarse en la investigación de los sistemas de GLN dentro del departamento de IA de MATCOM comenzó con la propuesta, en 2019, 
de realizar un modelo GLN capaz de dar cobertura a las actuaciones destacadas de los peloteros cubanos en las Grandes Ligas de Béisbol. La misma derivó 
en el trabajo de tesis de Roberto Balboa González \brackcite{balboa2020}. Este primer acercamiento a la GLN sirvió para buscar nuevos escenarios a abarcar 
desde el punto de vista de esta disciplina.\\


       \textbf{Problemática}\\

    Aún con el desarrollo de los sistemas de GLN y su mayor asimilación en diversos ámbitos, siguen siendo absolutamente predominantes los 
sistemas que tienen el inglés como idioma de referencia a la hora de generar el texto de salida. En la literatura consultada no abundan las soluciones 
en lenguaje español en este campo, entre otras razones influenciado por la gran complejidad estructural de este lenguaje, así como el menor número 
de herramientas específicas para este. De la misma forma esto impacta directamente en los sistemas que tienen como objetivo comunicativo los eventos deportivos. Siendo el deporte un objeto de mucho interés en la 
comunidad hispanohablante en general y en Cuba en particular, se hace necesario ampliar los escenarios existentes desde esta perspectiva.

    Los sistemas analizados en la literatura consultada en su mayoría se basan en un conocimiento explícito del dominio a tratar así como en una 
estructuración predefinida de los datos en base al dominio. A su vez, son muchas las propuestas que parten de la obtención de los datos desde 
la fuente como parte propia del sistema y no propiciados por el usuario. Son pocos los sistemas funcionales, fuera de la industria, capaces de 
desacoplarse de las fuentes de datos y de abarcar, desde una misma estructura de entrada de los datos, distintos dominios.\\

    %Los modelos neuronales están impactando el estado del arte de muchas tareas en el campo del PLN. Se hace necesario 
%acercar el prisma hacia los mismos desde el punto de vista de la GLN.\\

   

     %El resto de los antecedentes vienen enmarcados en el estudio de la literatura y de los sistemas existentes dentro del campo de la GLN que 
%se presentan en un capítulo posterior.%

%A pesar de los avances indiscutibles en el campo de la GLN queda aún mucho por hacer. Se considera que es interesante ampliar los sistemas que den cobertura informativa a los eventos 
%deportivos en idioma español. Además, es motivante crear un diseño de sistema para la generación de resúmenes textuales de enfrentamientos deportivos, que pueda servir dentro del 
%grupo de IA de MATCOM como antesala de futuras investigaciones y aplicaciones.


Teniendo en cuenta los antecedentes, las problemáticas, y partiendo de la hipótesis de que es posible definir un esquema general para representar los 
datos que describen los enfrentamientos deportivos, se arriba al objetivo del presente trabajo.\\

%se presenta un prototipo de sistema capaz de generar resúmenes de eventos deportivos a partir de los datos prove\'idos por 
%el usuario. Se utiliza el futbol y el boxeo como muestras para validar la propuesta. 

    %Un sistema de este tipo tiene diversos casos de uso, y entre otros, sirve para describir eventos de interés para un 
%determinado público que pudieran quedar fuera de la cobertura noticiosa de los medios de prensa. En cualquier otro escenario, 
%su utilidad queda marcada por las necesidades que quieran cubrir sus usuarios, con la única condición de poseer los datos.

    \textbf{Objetivo}\\

    Proponer el diseño de un sistema para la generación de res\'umenes o reportes de eventos deportivos 
independiente de la fuente de datos del dominio.
% utilizando como muestras para la validación el fútbol y el boxeo.
%A partir de dicho diseño se presentan dos modelos capaces de generar resúmenes de partidos de 
%fútbol y combates de boxeo. 


    Para la consecución del objetivo es necesario:

    \begin{itemize}
        %\item Estudiar la literatura relacionada con la GLN y determinar los enfoques factibles.
        \item Analizar el dominio y determinar características comunes y relevantes de los eventos deportivos.
        \item Crear un esquema general a partir del cual poder definir esquemas específicos para la entrada de datos de los distintos deportes.
        \item Proponer un diseño general para los modelos de generación de resúmenes.
        \item Validar la propuesta con dos modelos de generación de texto basados en reglas y plantillas para el fútbol y el boxeo.   
    \end{itemize}

    %\begin{itemize}
    %    \item Estudiar la literatura relacionada con la GLN y determinar los enfoques factibles. 
    %    \item Comprobar los sistemas de generación relacionados con el dominio (eventos deportivos) y sus caracter\'isticas.
    %    \item Estudiar el dominio, y determinar los rasgos distintivos y comunes entre los eventos deportivos.
    %    \item Diseñar un esquema de eventos comunes característicos entre los deportes que de lugar a un esquema específico de cada uno.
    %    \item Plantear una estructura intermedia para el ingreso de los datos en base al esquema general diseñado. 
    %    \item Diseñar e implementar los modelos generadores de texto.
    %    \item Presentar el sistema funcional para la generación de resúmenes.       
    %\end{itemize}

    \textbf{Estructura del trabajo}\\

    El resto del trabajo se organiza de la siguiente manera. El capítulo 1 aborda los problemas generales de los sistemas de GLN 
y las distintas técnicas relevantes en la literatura para su solución. En el capítulo 2 se presenta la propuesta de meta esquema para la 
representación de los datos de entrada, la metodología para definir los esquemas específicos, así como la propuesta de diseño para la 
general de los modelos de generación de resúmenes. En el capítulo 3 se presenta la validación
de la propuesta a través de la exposición de los esquemas de definición y los modelos generadores para el fútbol y el boxeo. 
Mientras en el capítulo 4 se analizan los detalles de la implementación del prototipo de sistema creado y se muestran los resultados 
textuales de los modelos tratados en el capítulo 3. El trabajo concluye con la presentación de las 
conclusiones y recomendaciones, así como la bibliografía consultada.