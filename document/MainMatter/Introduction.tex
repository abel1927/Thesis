\chapter*{Introducción}\label{chapter:introduction}
\addcontentsline{toc}{chapter}{Introducción}

    Desde hace varios años la Inteligencia Artificial(IA) viene revolucionando e \\impactando significativamente en muchas esferas de la 
vida del hombre. Dentro de la IA hay muchos campos de estudios y entre ellos el Procesamiento de Lenguaje Natural(PLN) es 
el encargado de lidiar con las problem\'aticas relacionadas con el lenguaje humano. Uno de los subcampos dentro del PLN que está 
acaparando especial atención en los últimos tiempos es el de la Generación de Lenguaje Natural(GLN) y en especial aquel que está 
enfocado a expresar a través de texto información contenida en forma de datos no lingüísticos(D2T por sus siglas en inglés, Data-to-Text). 
Existe consenso en la forma en que la salida de un sistema de GLN debe presentarse: texto, valga la redundancia, pero no hay un est\'andar 
en cuanto a la forma en que se presentan los datos para su procesamiento, variando de un sistema a otro. Como regla general el 
texto producido por estos sistemas debe mantenar fidelidad a los datos que lo originan, así como debe ser consecuente con su intención 
comunicativa, no siendo lo mismo un sistema para la generación de diálogos que uno que tiene como objetivo describir resúmenes biográficos.

    En un contexto donde la producción de datos se ha acelerado resultado de los avances tecnológicos y de los sistemas industriales, se ha hecho necesario para 
las empresas el manejo y la interpretación de estos datos. Empresas como \\ \textit{Automated Insights\footnote[1]{https://automatedinsights.com/}} 
se han especializado en la GLN y han enfocado su negocio en brindar soluciones a otras corporaciones con vista a automatizar sus procesos de producción de texto. 
Entre los casos de uso clásico de estas soluciones encontramos, en el marco del comercio electr\'onico, la generación de descripciones de productos a partir de sus 
fichas técnicas. El periodismo ha sido otra de las esferas beneficiadas de estos avances en la GLN. La generación automática de noticias(conocida como periodismo robótico) 
está cada vez más extendida, al punto de que importantes editoriales como \textit{The Washington Post} han creado sus propios sistemas para la generación de texto a partir 
de datos\footnote[2]{https://www.washingtonpost.com/pr/wp/2017/09/01/the-washington-post-leverages-heliograf-to-cover-high-school-football/}. 
En este caso su sistema se apoda \textit{Heliograf} y les permite cubrir todos los partidos de fútbol americano de las escuelas secundarias del área de Washington DC 
cada semana.

    Por su carácter estadístico y su gran audiencia, el deporte es una de las esferas que brinda diversas posibilidades para desarrollar sistemas 
de generación automática de texto. Es relativamente sencillo encontrar en la web los datos o tablas estadísticas de un determinado enfrementamiento deportivo, 
pero la gran variedad de eventos que se suceden constantemente hacen que sea humanamente imposible darle coberturas a cada uno de ellos a nivel de narración y 
resúmenes. A partir de lo cual surgen soluciones que cubren determinados eventos de interés como el ejemplo mencionado en el párrafo anterior.\\


   \textbf{Problema}\\

    Aún con el desarrollo de los sistemas de GLN y su mayor asimilación en diversos ámbitos, siguen siendo absolutamente predominantes los 
sistemas que tienen el inglés como principal o único lenguaje de referencia. El lenguaje español no presenta tanta cobertura en este 
campo. De la misma forma esto impacta directamente en los sistemas que tienen como objetivo comunicativo los eventos deportivos. Siendo el 
deporte un objeto de mucho interés en la comunidad hispanohablante en general y en Cuba en particular, se hace necesario ampliar los escenarios 
presentes desde esta perspectiva.

    Los sistemas presentes en la literatura en su mayoría se basan en un conocimiento explícito del dominio a tratar así como en una 
estructuración predefinida de los datos en base al dominio. A su vez, son muchas las propuestas que parten de la obtención de los datos desde 
la fuente como parte propia del sistema y no propiciados por el usuario. Son pocos los sistemas funcionales, fuera de la industria, capaces de 
desacoplarse de las fuentes de datos y de abarcar desde una misma estructura distintos dominios.

    Los recientes modelos neuronales están impactando el estado del arte de muchas tareas en el campo del PLN. Se hace necesario 
acercar el prisma hacia los mismos desde el punto de vista de la GLN.\\

    \textbf{Motivaciones}\\

    La propuesta del presente trabajo está motivada en las problemáticas anteriores. Siendo el deporte un campo que despierta tanto 
interés y que es fuente de entretenamiento de muchas personas, sentar las bases de una metodología para la generación de resúmenes de 
eventos deportivos es un reto estimulante.

    En el presente trabajo se presenta un sistema capaz de generar resúmenes de eventos deportivos a partir de los datos prove\'idos por 
el usuario. El mismo servirá, dado el caso, para dar descripción a eventos de interés de un determinado público que pudieran quedar fuera 
de la cobertura noticiosa de los medios de prensa.

    Así mismo este trabajo se presenta en el marco de las líneas de investigación existentes en el grupo de Inteligencia Artificial de la 
Facultad de Matemática y Computación de la Universidad de La Habana(MATCOM). Servirá como base para futuros trabajos que sigan ampliando y 
profundizando en la GLN.\\


    \textbf{Antecedentes}\\

    Esta vía investigativa dentro del departamento de IA de MATCOM comenzó con la propuesta en 2019 de realizar un modelo GLN capaz 
de dar cobertura a las actuaciones destacadas de los peloteros cubanos en la Grandes Ligas de Béisbol. Dicha propuesta derivó en el trabajo 
de tesis de Roberto Balboa González(~\cite{balboa2020}) que sirve de antecedente del presente trabajo.\\

    \textbf{Objetivo}\\

    El objetivo general de este trabajo es proponer una metodología funcional para la generación de res\'umenes de enfrentamientos deportivos 
independientes de la fuente de datos del dominio. A partir de dicha metodología se presenta la implementación de un sistema capaz de 
crear dichos resúmenes en idioma español. 
    Para la consecución del objetivo es necesario:

    \begin{itemize}
        \item Estudiar detalladamente la literatura relacionada con la GLN y determinar los enfoques factibles. 
        \item Comprobar los sistemas de generación relacionados con el dominio(enfrentamientos deportivos) y sus caracter\'isticas.
        \item Estudiar en profundidad el dominio, y determinar los rasgos distintivos y comunes entre los eventos deportivos.
        \item Diseñar un esquema de eventos comunes característicos entre los deportes que de lugar a un esquema específico de cada uno.
        \item Plantear una estructura intermedia para el ingreso de los datos en base al esquema general diseñado. 
        \item Diseñar e implementar los modelos generadores de texto.
        \item Presentar el sistema funcional para la generación de resúmenes.       
    \end{itemize}

    \textbf{Estructura del trabajo}\\

    El resto del trabajo se organiza de la siguiente manera. El capítulo 1 aborda los problemas generales de los sistemas de GLN 
y las distinas técnicas relevantes en la literatura para su solución. En el capítulo 2 se presenta el diseño del modelo a implementar, mientras 
que en el capítulo 3 se analizan los detalles de la implementación y los resultados. El trabajo concluye con la presentación de las 
conclusiones y recomendaciones para futuros trabjaos.