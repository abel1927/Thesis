\chapter*{Introducción}\label{chapter:introduction}
\addcontentsline{toc}{chapter}{Introducción}

    Desde hace varios años la Inteligencia Artificial (IA) viene revolucionando e \\impactando significativamente en muchas esferas de la 
vida del hombre. Dentro de la IA hay muchos campos de estudios y entre ellos el Procesamiento de Lenguaje Natural (PLN). Este abarca el conjunto de 
t\'ecnicas computacionales que tienen como objetivo el trabajo con lenguaje humano, incluyendo desde la extracci\'on de entidades en textos hasta modelos 
de comprensi\'on y generación de textos. Precisamente, uno de los subcampos dentro del PLN que está acaparando especial atención en 
los últimos tiempos es el de la Generación de Lenguaje Natural (GLN) y en especial aquel que está enfocado a expresar a través de texto 
información contenida en forma de datos no lingüísticos (D2T por sus siglas en inglés, Data-to-Text). Existe consenso en la forma en 
que la salida de un sistema de GLN debe presentarse: texto. Sin embargo no hay establecido un est\'andar en cuanto a la forma en 
que se presentan los datos para su procesamiento, variando de un sistema a otro. Como regla general el texto producido por estos sistemas 
debe mantenar fidelidad a los datos que lo originan, así como debe ser consecuente con su intención comunicativa, no siendo lo mismo un 
sistema para la generación de diálogos que uno que tiene como objetivo describir resúmenes biográficos.

    En un contexto donde la producción de datos se ha acelerado resultado de los avances tecnológicos y de los sistemas industriales, se ha hecho necesario para 
las empresas el manejo y la interpretación de estos datos. Por esa raz\'on hay empresas que comenzaron a brindar estos servicios a otras. Un ejemplo es la compañía 
\textit{Automated Insights\footnote[1]{https://automatedinsights.com/}} que ha enfocado su negocio en brindar soluciones a otras corporaciones con vista a automatizar sus 
procesos de producción de texto. Entre los casos de uso clásicos de estas soluciones encontramos, en el marco del comercio electr\'onico, la generación de descripciones de 
productos a partir de sus fichas técnicas. El periodismo ha sido otra de las esferas beneficiadas de estos avances en la GLN. La generación automática de noticias (conocida como 
periodismo robótico) está cada vez más extendida, al punto de que importantes editoriales como \textit{The Washington Post} han creado sus propios sistemas para la generación de 
texto a partir de datos\footnote[2]{https://www.washingtonpost.com/pr/wp/2017/09/01/the-washington-post-leverages-heliograf-to-cover-high-school-football/}. 
En este caso su sistema se apoda \textit{Heliograf} y les permite cubrir todos los partidos de fútbol americano de las escuelas secundarias del área de Washington DC 
cada semana.

    Por su carácter estadístico y su gran audiencia, el deporte es una de las esferas que se puede beneficiar claramente del desarrollo de los sistemas 
de generación automática de texto. Es relativamente sencillo encontrar en la web los datos o tablas estadísticas de un determinado enfrementamiento deportivo, 
pero la gran variedad de eventos que se suceden constantemente hacen que sea humanamente imposible darle cobertura a cada uno de ellos a nivel de narración y 
resumen. Por esta razón es importante el desarrollo de sistemas que sean capaz de cubrir muchas esferas del deporte.\\


   \textbf{Problema}\\

    Aún con el desarrollo de los sistemas de GLN y su mayor asimilación en diversos ámbitos, siguen siendo absolutamente predominantes los 
sistemas que tienen el inglés como idioma de referencia a la hora de generar el texto de la salida. En la literatura consultada no abundan las soluciones 
en lenguaje español en este campo, entre otras razones influenciado esto por la mayor complejidad estructural del español como lenguaje. De la misma forma esto 
impacta directamente en los sistemas que tienen como objetivo comunicativo los eventos deportivos. Siendo el deporte un objeto de mucho interés en la 
comunidad hispanohablante en general y en Cuba en particular, se hace necesario ampliar los escenarios existentes desde esta perspectiva.

    Los sistemas existentes en la literatura consultada en su mayoría se basan en un conocimiento explícito del dominio a tratar así como en una 
estructuración predefinida de los datos en base al dominio. A su vez, son muchas las propuestas que parten de la obtención de los datos desde 
la fuente como parte propia del sistema y no propiciados por el usuario. Son pocos los sistemas funcionales, fuera de la industria, capaces de 
desacoplarse de las fuentes de datos y de abarcar desde una misma estructura distintos dominios.

    Los modelos neuronales están impactando el estado del arte de muchas tareas en el campo del PLN. Se hace necesario 
acercar el prisma hacia los mismos desde el punto de vista de la GLN.\\

    \textbf{Motivaciones}\\

    La propuesta del presente trabajo está motivada en las problemáticas anteriores. Siendo el deporte un campo que despierta tanto 
interés y que es fuente de entretenimiento de muchas personas, sentar las bases de una metodología para la generación de resúmenes de 
eventos deportivos es un reto estimulante.

    En el presente trabajo se presenta un sistema capaz de generar resúmenes de eventos deportivos a partir de los datos prove\'idos por 
el usuario. El mismo tiene diversos casos de uso, y entre otros, servirá para dar descripción a eventos de interés de un 
determinado público que pudieran quedar fuera de la cobertura noticiosa de los medios de prensa. En cualquier escenario su utilidad quedará 
marcada por las necesidad que quieran cubrir sus usurios, con la única condición de poseer los datos.

    Así mismo este trabajo se presenta en el marco de las líneas de investigación existentes en el grupo de Inteligencia Artificial de la 
Facultad de Matemática y Computación de la Universidad de La Habana(MATCOM). Servirá como base para futuros trabajos que sigan ampliando y 
profundizando en la GLN.\\


    \textbf{Antecedentes}\\

    La intención de adentrarse en la investigación de los sitemas de GLN dentro del departamento de IA de MATCOM comenzó con la propuesta, en 2019, 
de realizar un modelo GLN capaz de dar cobertura a las actuaciones destacadas de los peloteros cubanos en la Grandes Ligas de Béisbol. La misma derivó 
en el trabajo de tesis de Roberto Balboa González (~\cite{balboa2020}). Este primer acercamiento a la GLN sirvió para buscar nuevos escenarios a abracar 
desde el punto de vista de esta disciplina.

    El resto de antecedentes de este trabajo vienen enmarcados en el estudio de la literatura y de los sitemas existentes dentro del campo de la GLN que 
se presentan en un capítulo posterior.\\

    \textbf{Objetivo}\\

    El objetivo general de este trabajo es proponer una metodología funcional para la generación de res\'umenes de eventos deportivos 
independientes de la fuente de datos del dominio. A partir de dicha metodología se presenta la implementación de un sistema capaz de 
crear dichos resúmenes en idioma español. 
    Para la consecución del objetivo es necesario:

    \begin{itemize}
        \item Estudiar detalladamente la literatura relacionada con la GLN y determinar los enfoques factibles. 
        \item Comprobar los sistemas de generación relacionados con el dominio (eventos deportivos) y sus caracter\'isticas.
        \item Estudiar en profundidad el dominio, y determinar los rasgos distintivos y comunes entre los eventos deportivos.
        \item Diseñar un esquema de eventos comunes característicos entre los deportes que de lugar a un esquema específico de cada uno.
        \item Plantear una estructura intermedia para el ingreso de los datos en base al esquema general diseñado. 
        \item Diseñar e implementar los modelos generadores de texto.
        \item Presentar el sistema funcional para la generación de resúmenes.       
    \end{itemize}

    \textbf{Estructura del trabajo}\\

    El resto del trabajo se organiza de la siguiente manera. El capítulo 1 aborda los problemas generales de los sistemas de GLN 
y las distinas técnicas relevantes en la literatura para su solución. En el capítulo 2 se presenta el diseño del modelo a implementar, mientras 
que en el capítulo 3 se analizan los detalles de la implementación y los resultados. El trabajo concluye con la presentación de las 
conclusiones y recomendaciones para futuros trabjos.