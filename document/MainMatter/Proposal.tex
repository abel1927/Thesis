\chapter{Propuesta}\label{chapter:proposal}
 
    El objetivo de desarrollar un sistema para generar resúmenes de eventos deportivos independiente de la fuente de datos 
planteó distintos retos. El primero de estos fue la necesidad de definir un esquema que permitiera abstraer las características 
generales del conjunto de deportes de enfrentamiento (enfrentamiento dos a dos). Junto con dicho esquema se necesitó 
definir una estructura común para la entrada de los datos que permitiera expresar los conocimientos del dominio. Se seleccionó 
una estructura basada en tuplas de cuatro elementos (cuatro-tuplas en lo adelante). A partir de esta estructura y en base 
al esquema de definición general se buscó poder determinar esquemas específicos para cada deporte que se fuera a incluir en el 
sistema. Cada uno de estos esquemas específicos son los que se encargan de definir un deporte de forma individual dentro del sistema.
    
    La segunda parte del trabajo consistió en construir en base a la propuesta, un esquema y su modelo correspondiente para generar 
resúmenes de partidos de fútbol. Lo mismo se realizó con un deporte de naturaleza diferente como el boxeo.

\section{Propuesta de Esquema General}

    Los deportes se pueden clasificar en la categoría de individuales o colectivos. En los deportes colectivos, las representaciones 
del enfrentamiento ocurren en base a equipos que agrupan a individuos. A su vez, en los deportes individuales son dos los contendientes.
Esta es la primera diferencia que se extrae en el análisis del conjunto de deportes. Las modalidades analizadas están representadas en \ref{tab:table_deportes_seleccionados}, 
clasificadas en individuales o colectivas.

% Please add the following required packages to your document preamble:
% \usepackage{multirow}
\begin{table}[]
    \begin{center}
    \begin{tabular}{|c|c|}
    \hline
    Colectivos                                                                                                                                          & Individuales                                                                                                     \\ \hline
    \multirow{5}{*}{\begin{tabular}[c]{@{}c@{}}Béisbol, Voleibol, \\ Fútbol, Tenis Dobles, \\ Baloncesto, Waterpolo, \\ Balonmano, Hockey\end{tabular}} & \multirow{5}{*}{\begin{tabular}[c]{@{}c@{}}Tenis, Esgrima, \\ Boxeo, Judo,\\ Lucha libre, Taekwondo\end{tabular}} \\
                                                                                                                                                        &                                                                                                                  \\
                                                                                                                                                        &                                                                                                                  \\
                                                                                                                                                        &                                                                                                                  \\
                                                                                                                                                        &                                                                                                                  \\ \hline
    \end{tabular}
    \caption{Deportes analizados}
    \label{tab:table_deportes_seleccionados}
    \end{center}
\end{table}

    De cada uno de estos deportes se analizó:

    \begin{itemize}
        \item Naturaleza de decisión: La mayoría de los deportes se definen como juegos 
        adversariales por acumulación de puntos. La entidad con mayor puntuación gana. Otros, como el tenis y el voleibol 
        se definen por cantidad de etapas ganadas (sets), y cada etapa se gana por puntos. A su vez, en el boxeo la definición 
        se deriva de votaciones de árbitros.
        \item Posibilidad de empate: Hay deportes como el fútbol en el que, según la competición o la fase de ésta, existe la posibilidad de 
        definirse sin ganadores ni perdedores.
        \item División de los eventos: La mayoría de los eventos se divide por etapas de tiempo.
        constante. Una excepción es el judo que ocurre de forma continua durante cuatro minutos. 
        \item Alargues de tiempo: La mayoría de los deportes, en caso de no definición en su tiempo reglamentario, presentan 
        etapas adicionales en forma punto de oro (ej. judo),  tiempos extras (ej. béisbol, fútbol), tiebreak (desempate, ej. voleibol, tenis).
        \item Roles: Dentro de los deportes los participantes ejercen roles, como puede ser su posición en los deportes colectivos. En los deportes 
        individuales estos roles no son tan explícitos.
        \item  Acciones principales: La definición de los eventos son las acciones relevantes que ocurren durante el tiempo de juego.
    \end{itemize}

    Del análisis también se extrajeron un conjunto de características que son comunes a los enfrentamientos deportivos: la sede, 
el público, la fecha. Asimismo, los enfrentamientos normalmente se encuadran dentro de un torneo, y existen distinciones entre categorías lo mismo sea 
de edad, sexo, u de otro tipo (ej. peso).

    A partir del análisis se definió un meta esquema general de tipos de entradas basado en una estructura de 
cuatro-tuplas de conocimiento.

\begin{table}[]
    \begin{center}

\begin{tabular}{|c|c|}
    \hline
    Tipo de Entrada  & Estructura                                                                                                               \\ \hline
    SEDE             & \begin{tabular}[c]{@{}c@{}}(TipoSEDE, \\ Nombre\\ Asistencia\\ Capacidad)\end{tabular}                                   \\ \hline
    TORNEO           & \begin{tabular}[c]{@{}c@{}}(TipoTORNEO\\  Nombre\\ Expresión de Género\\  Expresión de Categoría)\end{tabular}           \\ \hline
    ENFRENTAMIENTO   & \begin{tabular}[c]{@{}c@{}}(TipoENFRENTAMIENTO\\ Entidad\_1\\  Entidad\_2\\  Expresión de Fecha)\end{tabular}            \\ \hline
    ROLENJUEGO       & \begin{tabular}[c]{@{}c@{}}(TipoROLENJUEGO\\  Entidad del Rol\\  Entidad Complementaria\\ Rol Complementario)\end{tabular}   \\ \hline
    RESULTADOPARCIAL & \begin{tabular}[c]{@{}c@{}}(TipoRESULTADOPARCIAL\\ Entidad\\ Indicador de parcial\\ Expresión de puntuación)\end{tabular}    \\ \hline
    RESULTADOFINAL   & \begin{tabular}[c]{@{}c@{}}(TipoRESULTADOFINAL\\ Entidad\\ Expresión de puntuación\\  Descriptor de resultado)\end{tabular}  \\ \hline
    EVENTO           & \begin{tabular}[c]{@{}c@{}}(TipoEVENTO\\  Expresión de Tiempo\\ Entidad Protagonista\\  Entidad Complementaria)\end{tabular} \\ \hline
    \end{tabular}
        
    \end{center}
    \caption{Meta esquema general para definir las entradas de cada deporte}
    \label{tab:esquema_general}
\end{table}

    Cada cuatro-tupla tiene en la primera posición el tipo de entrada. El resto de los valores constituyen la base de información. Con cada tipo de entrada 
se encapsula un subconjunto de la información que se muestra, común al conjunto de deportes estudiados. 
    A partir del meta esquema general, es posible definir los esquemas específicos de cada deporte, con sus tipos particulares para cada entrada y su forma de interpretar 
cada uno de los valores. Los esquemas de cada deporte tienen que ser capaces de poder expresar la información del mismo para poder, a través de ella, 
generar textos que describan el enfrentamiento. \\

%\pagebreak

    \textbf{Ejemplos abstractos de formación de entradas}\\

    Se presenta una meta representación de entradas basadas en el esquema general y su interpretación en el contexto del sistema.

    \begin{itemize}
        \item (SEDE, A, 1450, 1700) : El enfrentamiento ocurre en la sede de nombre A, con capacidad para 1700 espectadores, asistieron 1450.
        \item (TORNEO, B, F, categoría\_1) : El enfrentamiento pertenece al torneo B, femenino, en la categoría categoría\_1.
        \item (ENFRENTAMIENTO, contrincante\_A, contrincante\_B, 11-11-2022) : Se enfrentan contrincante\_A y contrincante\_B el 11 de noviembre de 2022. 
        \item (ROLENJUEGO, individuo\_A, entidad\_A, segundo\_rol) : El individuo\_A desarrolla primer\_rol y segundo\_rol respecto a entidad\_A.
        \item RESULTADOPARCIAL:
            \begin{itemize}
                \item (RESULTADOPARCIAL, contrincante\_A , P, X): En el parcial P, contrincante\_A tiene X puntos.
                \item (RESULTADOPARCIAL, contrincante\_B , P, Y): En el parcial P, contrincante\_B tiene Y puntos.
            \end{itemize}
        \item RESULTADOFINAL:
            \begin{itemize}
                \item (RESULTADOFINAL, contrincante\_A, X, Derrota): El contrincante\_A perdió con X puntos.
                \item (RESULTADOFINAL, contrincante\_B, Y, Victoria): El contrincante\_B ganó con Y puntos.
            \end{itemize}
        \item EVENTO: 
            \begin{itemize}
                \item (EVENTO, tiempo\_x , individuo\_A, “”): En el tiempo\_x, el individuo\_A protagonizó el EVENTO
                \item (EVENTO, tiempo\_y, individuo\_A, individuo\_B): En el tiempo\_y, individuo\_A protagonizó el EVENTO en 
                complemento de (en oposición de, en beneficio de, en perjuicio de, en relación con, respecto a) individuo\_B. 
            \end{itemize}
    \end{itemize}

   

%A continuación se presenta la definición para dos deportes de naturaleza y estructura muy dispar: el fútbol y 
%el boxeo.

\section{Metodología para la conformación de los esquemas específicos}

    Primero, es necesario tener en cuenta que el meta esquema planteado anteriormente busca la abstracción de conceptos comunes. Estos 
conceptos necesitan, al menos los referentes a los roles, eventos y resultados, ser llevados a su expresión específica dentro de una modalidad 
deportiva. A su vez, se deben diferenciar los conceptos de: capacidad de representación y obligación de representación. Que el esquema permita definir 
un tipo de información determinada no significa que todos los deportes expresen necesariamente ese concepto. Además es posible que se conciban distintos 
esquemas específicos para un mismo deporte. Esto depende de cómo cada modelo sea representado.

    La representación de un deporte en un esquema específico a partir del esquema general propuesto necesita del análisis de sus características. 
Se debe realizar un estudio que permita detallar las situaciones que presenta el deporte y representarlas basado en eventos. A partir de 
estudiar las reglamentaciones, se separa al deporte en cuanto a su categoría: individual o colectivo. 

    Para los deportes colectivos la expresión de los roles de los deportistas se encuentra mínimamente definida a partir del concepto de 
alineación. Esta serie de deporte tiene un conjunto de individuos que inician las disputas de los encuentros y otros que ingresan a raíz de decisiones 
que se toman durante el transcurso del evento. Además, se pueden expresar conceptos como las disposiciones que ocupa cada deportista dentro del equipo. En este 
tipo de información, la \textit{entidad complementaria} que define la tupla de \textit{ROLENJUEGO} sería el equipo del deportista.
    En el caso de los deportes individuales, los roles no se expresan tan claramente. Aun así, es posible identificar roles de representación, ya sea de un 
país, una delegación, un equipo multi categoría. Un ejemplo fuera de los deportes de enfrentamiento se encuentra en la fórmula 1, donde los competidores 
representan a escuderías durante las carreras.

    En lo referido a los parciales, se necesita determinar las etapas en las que trascurre un enfrentamiento en caso de que este ocurra por etapas. La información 
de los parciales permite al sistema desambiguar situaciones que ocurren durante los enfrentamientos, así como da la posibilidad de dotar de más información 
la narrativa. Para conformar las tuplas de \textit{RESULTADOPARCIAL}, es necesario determinar si existe uno o m\'as tipos de segmentación dentro del enfrentamiento, así como 
lograr una expresión identificativa que sea única para cada una.

    La expresión de los eventos es la que dota principalmente de capacidad descriptiva a los modelos. Los eventos, acciones que se suceden en un deporte, son la esencia de 
este y por esa razón son la información fundamental que sobre ellos se transmite, más allá del resultado. Para expresar los eventos es necesario en primera instancia determinar cuáles 
son los que existen dentro de la modalidad seleccionada. A partir de esto, definir si para su expresión es necesario el concepto de antagonista como sujeto no protagonista en la acción.
También se debe determinar una expresión temporal que identifique de forma única y cronológica la secuencia de eventos. De esta forma, no se generan ambigüedades a la hora de que los 
modelos interpreten los mismos.

    Los datos referentes a la sede, el público, el torneo, el resultado y las categorías se expresan de forma más directa. Queda en decisión del realizador del esquema y su modelo específico, 
determinar qué informaciones constituyen un requerimiento en el contexto de la generación del resumen y cuáles son complementos informativos. Es decir, el modelo sería capaz de lidiar con 
la ausencia de determinados datos. 



