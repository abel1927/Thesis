\begin{opinion}
    
    El estudiante Abel Molina Sánchez desarrolló satisfactoriamente el trabajo de diploma 
titulado “Generación automática de reportes textuales sobre enfrentamientos deportivos”. En este 
trabajo el estudiante propuso el diseño de un sistema para la generación automática de notas 
textuales sobre eventos deportivos. La idea general seguida fue la definición de un esquema general, 
o meta-esquema, de tuplas (4-tuplas) de conocimiento sobre las cuales se pudieran implementar, 
siguiendo una metodología genérica, funciones de realización particulares. 
    
    Para validar esta propuesta, el estudiante propuso esquemas particulares para dos deportes con 
características diferentes: el fútbol y el boxeo. Además, propuso un conjunto de funciones de 
realización lingüística para cada caso. Con ello pudo mostrar como se generaban distintos reportes 
para bases de conocimiento diferentes y así mostrar la validez y factibilidad de la propuesta.
    
    Para poder afrontar el trabajo, el estudiante tuvo que revisar literatura científica relacionada 
con la temática así como soluciones existentes y bibliotecas de software que pueden ser apropiadas 
para su utilización. Todo ello con sentido crítico, determinando las mejores aproximaciones y también 
las dificultades que presentan.
    
    Todo el trabajo fue realizado por el estudiante con una elevada constancia, capacidad de trabajo y 
habilidades, tanto de gestión, como de desarrollo y de investigación. 
    
    Por estas razones pedimos que le sea otorgada al estudiante Abel Molina Sánchez la máxima calificación 
y, de esta manera, pueda obtener el título de Licenciado en Ciencia de la Computación.
    

\vspace{1cm}
\begin{flushright}
    \emph{Dr. Yudivián Almeida Cruz} \\
    %Facultad de Matemática y Computación \\
    %Universidad de La Habana    
\end{flushright}   


%Dr. Yudivián Almeida Cruz

\end{opinion}